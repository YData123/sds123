
%\input{$HOME/YData/lectures/ydata_lecture_styles}
\input{../../lectures/ydata_lecture_styles}

%%%TITLE SLIDE (defined in style file)
\tslide{11}{Joins}


%%%SLIDE
\frame{
\begin{center}
\huge \tt Announcements
\end{center}
}


%%%SLIDE
\frame{
\begin{center}
\huge \tt  Project 1: World Progress
\end{center}
\vfill

{\footnotesize Video: \url{www.gapminder.org/videos/dont-panic-the-facts-about-population/}}
}



%%%SLIDE
\frame{
\begin{center}
\huge \tt Joins
\end{center}
}


%%%SLIDE
\frame{
{Joining Two Tables}

\begin{center}
\includegraphics[width=0.99\textwidth]{joins_drinks.png} 
\end{center}

\vfill
\begin{center}
\alert{(DEMO)}
\end{center}
}


%%%SLIDE
\frame{
\begin{center}
\huge \tt Bikes
\end{center}
\vfill
\begin{center}
\alert{(DEMO)}
\end{center}
}


%%%SLIDE
\frame{
\begin{center}
\huge \tt Shortest Trips
\end{center}
\vfill
\begin{center}
\alert{(DEMO)}
\end{center}
}



%%%SLIDE
\frame{
\begin{center}
\huge \tt Maps
\end{center}
\vfill
\begin{center}
\alert{(DEMO)}
\end{center}
}



%%%SLIDE
\frame{
{Maps}
A table containing columns of latitude and longitude values can be used to generate a map of markers
\bigskip

\begin{center}
\includegraphics[width=0.99\textwidth]{map_table.png} 
\end{center}
}


%\frame{ 
%{References} \footnotesize
%\bibliographystyle{$HOME/YData/bibliography/asa}
%\bibliography{$HOME/YData/bibliography/ydata_bibliography}
%}
\end{document}
